\documentclass[UTF8]{ctexart}
\title{迟到指南}
\author{未知艺术家}
\date{2021/08/31/19/53/08}
\begin{document}
	\maketitle
	\tableofcontents
	\section{综述}
同学们在高三时期经常会熬夜学习,而熬夜学习的必然结果是早晨起不来从而导致早晨迟到。\\
迟到者通常会被以李某为首的学生会所围剿,而这显然是广大劳苦普通学生所不愿遇到的。\\
本文作者拥有多年的迟到经验,了解学生会的种种侦察手段以及如何巧妙化解这些手段。\\
《礼记》有言:“凡事豫则立,不豫则废。”只有提前做好心理建设,才能打赢这场没有硝烟的战争!\\
因此,我将从以下几个方面教大家如何请假以及如何在迟到时应对学生会的围剿:\\
\begin{itemize}
	\item 常见失败请假借口示例以及失败原因分析
	\item 如何正确的请假
	\item 学生会常见的侦察手段以及如何反侦察
\end{itemize}

\maketitle
	\section{常见失败请假借口示例以及失败原因分析}
	\subsection{常见失败请假借口示例}
	\begin{table}[h!]
		\begin{center}
			\caption{常见失败请假借口示例}
			\begin{tabular}{l|c|r} % <-- Alignments: 1st column left, 2nd middle and 3rd right, with vertical lines in between
				\textbf{指数排名} & \textbf{名称} & \textbf{指数}\\
				$ExponentyRanking$ & $Name$ & $Exponent$ \\
				\hline
				1 & 亲人反复去世 & 100.00\\
				2 & 表姐反复出嫁 & 20.36\\
				3 & 妈妈二胎陪产 & 19.35\\
				4 & 朋友得白血病 & 2.91\\
				5 & 陪闺蜜堕胎 & 1.94\\
				6 & 出席亲妈婚礼 & 1.93\\
				1 & 父母外出,怕宠物狗饿死 & 9.38\\
				2 & 母猪难产 & 9.08\\
				1 & 家里天花板塌了 & 36.04\\
				2 & 反复痔疮手术 & 30.55\\
				3 & 割包皮 & 16.91\\
				4 & 吃苹果吃过敏了 & 8.63\\
			\end{tabular}
		\end{center}
	\end{table}
\newpage
	\subsection{数据来源}
数据来源:提取知乎“你听过哪些请假奇葩理由”“有什么好的请假借口让老板无法拒绝”等话题内容,截止时间为2021年5月17日。\\
\textit{注:指数计算基于相关样本分词结果,以词频最高的为基准指数100。}
	\subsection{失败原因分析}
	这些借口实在是太陈旧了,\textbf{请假借口要做到“经新”。}\\
	这些招数,甚至打破了国界,做到了世界通用。\\
	东康涅狄格州立大学的学者将这种流行在美国大学生中的特殊病症,称为“重病祖母综合征”,只要一到考试前,学生的祖母们突发急症的概率要比平常高得多!\\
	难道你以为学生会真的有这么傻吗?!\\
	\textit{参考文献:Adams, M. (1990). The dead grandmother/exam syndrome and the potential downfall of American society. The Connecticut Review.}
\section{如何正确的请假}
\paragraph{只要演技到位,这些病假借口真的很难证伪!}
	\begin{table}[h!]
	\begin{center}
		\caption{很难证伪的病假借口示例}
		\begin{tabular}{l|c|r}% <-- Alignments: 1st column left, 2nd middle and 3rd right, with vertical lines in between
			\textbf{指数} & \textbf{理由} & \textbf{示例/操作技巧}\\
			$Exponent$ & $Name$ & $Example$ \\
			\hline
			100.00 & 肚子疼/拉肚子 & 详见下文\\
			57.89 & 发烧 & 详见下文\\
			56.73 & 感冒 & 只需—张温度计图片!\\
			45.03 & 痛经 & 男老师不好意思问太清楚。\\
			31.58 & 痔疮 & 学生会应该不会让你脱裤子检查吧。\\
			16.96 & 头痛/头晕 & 这个不需要报告单,也没办法查。\\
			8.19 & 咳嗽 & 突然发现发烧以及咳嗽流鼻涕了,想在家自我观察一天。\\
			7.02 & 割包皮 & 详见下文\\
		\end{tabular}
	\end{center}
\end{table}

	\subsection{数据来源}
数据来源:提取豆瓣小组“请假借口研究所”、知乎和微博“请假理由”“请假借口”等相关话题内容并进行分词,截止时间为2021年5月17日。\\
\textit{注:指数计算基于相关样本分词结果,以词频最高的为基准指数100。}
\newpage
	\subsection{具体操作技巧}
\paragraph{使用【肚子疼/拉肚子】请假的操作技巧}
每次都装肚子疼,先去厕所催吐一下,然后面色苍白,不要舔自己的嘴唇,让它很干,然后去找老师打请假条,还要装出病怏快的样子。 
\paragraph{使用【发烧】请假的操作技巧}
发烧理由不奇葩但是操作很奇葩:①搓手搓到热,然后捂脑门。②反复搓热捂脑门。既保证脑门热,又不会留下搓痕,然后顺理成章的在家歇病假。
\paragraph{使用【割包皮】请假的操作技巧}
此方法对于女老师来说百分百成功!对于男老师:排除5\%的变态,也很稳!


\section{学生会常见的侦察手段以及如何反侦察}
	\subsection{常见的侦察手段}
	\paragraph{较易识破的}
	\begin{itemize}
		\item 【甜言蜜语型】:“同学你过来一下哦~没事的~就是记个名字而已的”
		\item 【威逼利诱型】:我是某某某,你赶紧过来,要不然我就去告诉政教处!
		\item 【直接粗暴型】:这种人会直接拦住你,并且对你死缠烂打。
	\end{itemize}
	\paragraph{较难识破的}
\begin{itemize}
	\item 【间谍型】:据作者对多名劳苦普通学生的采访,其中很多人都经历过间谍式的侦察,下面举一例子:\\
			\textit{例:某天早晨,作者正在无学生会成员的小路上匍匐前进之时,突然有一名拿着扫帚的高一新生跑来,作者以多年的反学生会侦察经验立即反应过来:“这是学生会的间谍!”果不其然,这名间谍在看到作者不到5秒钟就开始大声呼喊其学生会的同僚。而作者则因为经验丰富而逃之夭夭。}
	\item 【游击战术型】:有时学生会的游击队员兼侦察员会在早晨隐藏在教学楼的各处,这种侦察手段一定要严加防范!
\end{itemize}
	\paragraph{较易识破但较难应对的}
		\begin{itemize}
		\item 【学生会成员围剿型】:即多名学生会团伙骨干分子对迟到者进行围剿的情况
		\item 【威逼利诱+教师围剿型】:即一名或多名学生会团伙骨干分子对迟到者采取心理战术的情况\\
		\textit{例:你要是告诉我你叫什么名字就没事,要是不说我就去找政教处某某某,你就完了!}
		\end{itemize}
	\subsection{常见的反侦察方法}
	\subsubsection{甜言蜜语型}
	[待编写]
	\subsubsection{威逼利诱型}
	[待编写]
	\subsubsection{直接粗暴型}
	[待编写]
	\subsubsection{间谍型}
	[待编写]
	\subsubsection{游击战术型}
	[待编写]
	\subsubsection{学生会成员围剿型}
	[待编写]
	\subsubsection{威逼利诱+教师围剿型}
	[待编写]
		\subsection{常见的应对方法}
		\textit{当反侦察方法失败了,我们要如何应对学生会的围剿呢?\\
		在本节中,我将分类介绍如何应对学生会的围剿}
\subsubsection{甜言蜜语型}
[待编写]
\subsubsection{威逼利诱型}
[待编写]
\subsubsection{直接粗暴型}
[待编写]
\subsubsection{间谍型}
[待编写]
\subsubsection{游击战术型}
[待编写]
\subsubsection{学生会成员围剿型}
[待编写]
\subsubsection{威逼利诱+教师围剿型}
[待编写]

\section{编后记}
	[待编写]
\section{致谢}
巫雨松  \textit{网易新闻中心编辑}\\
\dag 本文使用\LaTeX 排版系统排版
\footnote{\ddag\LaTeX is a document preparation system used for the communication and publication of scientific documents.}
\end{document}